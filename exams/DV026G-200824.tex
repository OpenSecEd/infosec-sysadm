% Author:  Daniel Bosk <daniel.bosk@miun.se>
\documentclass{article}
\usepackage[utf8]{inputenc}
\usepackage[T1]{fontenc}
\usepackage[swedish,english]{babel}
\usepackage[hyphens]{url}
\usepackage{hyperref}
\usepackage{color}
\usepackage{prettyref,varioref}
\usepackage{subfigure}
\usepackage{amsmath,amssymb}
\usepackage{listings}
\usepackage{authblk}
\usepackage[all]{foreign}
\usepackage{enumitem}

\usepackage{acro}
\DeclareAcronym{ILO}{%
  short = ILO,
  short-indefinite = an,
  long = intended learning outcome,
  long-indefinite = an,
}

\usepackage{csquotes}
\MakeBlockQuote{<}{|}{>}
\EnableQuotes

\usepackage[natbib,style=alphabetic,maxbibnames=99]{biblatex}
\addbibresource{literature.bib}

%\printanswers

%\courseid{DV026G}
%\course{Information Security}
%\assignmenttype{Final exam}
\title{Exam in DV026G Information Security: A small case}
\date{2020-08-24}
\author{%
  Daniel Bosk
}
\affil{%
  Department of Information Systems and Technology,\\
  Mid Sweden University, SE-851\,70 Sundsvall\\
  Email: \href{mailto:daniel.bosk@miun.se}{daniel.bosk@miun.se}\\
  Phone: 010-142\,8709
}

\begin{document}
\maketitle

\section*{Instructions}%
\label{sec:Instructions}
You will be given a small assignment.
You should solve this assignment in groups, with no limit on resources.
The group will write up the solution in \emph{one report per group} and hand it 
in before the deadline.

To provide a \enquote{proof of knowledge}, everyone will book a timeslot for a 
15-minute individual meeting in Zoom.
During this meeting I will ask you
\begin{enumerate}
  \item if there's anything you'd like to change in the solution (you might not 
    agree fully with the group);
  \item about some selected details of the solution.
\end{enumerate}
Note that during this meeting you \emph{must have a working webcam and 
microphone and a valid ID}.

I strongly advice you to ensure that everyone is involved in every part of the 
report.
For instance, write the report in a joint Google Docs or Overleaf document 
while everyone participates over Zoom (or similar).
(To make it easier for everyone to navigate the editing, it might be good to 
name someone secretary and that person shares the screen.
The secretary position doesn't have to be fixed, it can rotate every now and 
then.)

Also, write your own notes about various design decisions during the work in 
the group, particularly those that you disagree with.


\section*{Grading criteria}

The relevant \acp{ILO} are: that you are able to
\begin{itemize}
  \item \emph{apply} different cryptographic primitives and \emph{explain} how 
    these work (on a high level);

  \item \emph{analyse} problems of authentication, access control, 
    accountability and different solutions;

  \item \emph{explain} how some common attacks on software works and 
    \emph{analyse} code for security vulnerabilities;

  \item \emph{evaluate} strengths and weaknesses of hardware-based security 
    such as full-disk encryption.
\end{itemize}

The grades will be based on the following grading criteria.
\begin{description}
  \item[Grade E] You fulfil all the \acp{ILO} above.
    You should have identified a relevant problem, and given a solution to it.
    It must be a viable solution, however gaps and mistakes are allowed, if 
    they don't render your solution unusable.

  \item[Grade C] You fulfil the criteria for E.
    Additionally, your evaluations and designs are \emph{good} with \emph{some 
      base} in theory and, where applicable, the research literature.
    Gaps and errors are allowed if they only render your solution less optimal.

  \item[Grade A] You fulfil the criteria for C.
    However, your evaluations and designs must be \emph{extensive} (in detail) 
    and \emph{well-founded} in theory and, where applicable, the research 
    literature.
    Gaps and errors are not allowed in the solution unless they have been 
    properly addressed and you have given a suggestion for an approach to how 
    to start resolve the issue.
\end{description}
The grades B and D are intermediary grades.


\clearpage
\section*{The assignment}

You should design the architecture for a digital post service.
This means a high-level sketch of a protocol, the outline of the app internals 
and the server-side API\@.

The app should provide the following features:
\begin{enumerate}
  \item A user can register for a post box.
    As this post service will replace physical mail, it must be tied to a real 
    identity.

  \item The owner can read the mail.
    Only the owner should be able to do this --- no one else, not even service 
    staff!

  \item A sender should be able to send mail to a recipient.
    This mail should not be readable by any intermediary.

  \item Anonymous mail.
    A sender can send a letter to a recipient, but neither the recipient nor 
    the service will learn the identity of the sender.
    The service will not learn the recipient either.
\end{enumerate}
You can see the last one as optional for a higher grade.

Solving this assignment will touch upon almost every topic in the course.
Make sure to base your designs on the theory of the course, add references, 
that will help you (remember that's a part of the grading criteria).
\enquote{This feels secure} is not a convincing argument.
Likewise, \enquote{all connections should use TLS} will not cut it either; why 
do you want TLS, what properties do you need and which of those will TLS 
provide and why?
The same applies for BankID, what properties are you after?
How does BankID provide those and what are the limitations?




\printbibliography
\end{document}

