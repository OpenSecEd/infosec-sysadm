\question[3]
% tags: auth:E:C:A
% tags: accountability:E:C:A
% tags: usability:E:C:A
There is a strong relation between accountability and identity (why?).
There is also a strong relation between identity and authentication (why?).
Finally, there is a strong relation between authentication and usability (why?).
Discuss the various requirements and possible trade-offs needed to get proper 
accountability.
(Also answer the why-questions above.)

\begin{solution}
Accountability needs the identity of the subject to hold the subject accountable 
for its actions.
To ensure that the identity of the subject is correct, we need authentication to 
authenticate the subject's identity.
Authenticating the subject of course affects the usability.

Accountability is long term, concerning many events over maybe days, weeks or 
years.
This means that we need to authenticate every (critical) action of the subject.
This affects usability even more (than just authenticate at login, for 
instance).
\end{solution}
\question[3]
% tags: ac
% tags: E:C:A
What is mandatory access control?
Discuss its advantages and disadvantages and its suitability in different 
situations.

\begin{solution}
  Mandatory access control sets the access policy for created objects based on 
  fixed rules in the system.

  This helps enforcing the policy and avoid mistakes.
  On the other hand, it will prevent communication downwards, which might be 
  necessary sometimes (this is usaully solved by special procedures for 
  declassification).

  However, whether it is good or bad depends on the situation.
  Sometimes it's good to have a combination of mandatory and discretionary 
  access control.
\end{solution}


\question[3]
% tags: foundations:E:C:A

Discuss the overall goal of the security field and its role in society.

\begin{solution}
The goal of the security field is to provide means to prevent, detect or recover 
from failures.
This includes all kinds of systems, from buildings to smartphones.

Security must take a more central role in society as society becomes more and 
more dependent on digital systems whose interfaces are available to everyone, 
e.g.\ through the Internet --- for example, a web portal to the company network 
or the interface of a web-connected surveillance camera or \enquote{Internet of 
Things} in general.

It must be everyone's responsibility to improve security --- no one must neglect 
it!
This was clearly illustrated by the Mirai botnet, where vulnerabilities in 
ordinary peoples' IoT products (webcams, video recorders, etc.) were used to 
take down large parts of the core Internet infrastructure.
Although a vulnerability in a webcam intuitively is only bad for the person 
having the webcam --- some remote person can spy on them, record secret videos 
etc.\ --- this is not the case, it can be bad for everyone as that webcam can be 
used to execute attacks against others.
Thus one insecurity \enquote{here} might actually result in insecurities 
\enquote{there}, \enquote{there} and \enquote{there}.
It is very hard to predict.
\end{solution}
 
\question[3]
% tags: passwd:E:C:A
% tags: infotheory:E:C:A
You've just landed a job at an IT department somewhere and now you're having one 
of your first few days.
There is a discussion in the \enquote{fika room}, the topic is the IT 
department's password policy.

\enquote{Well, every respectable website requires at least eight characters, 
with lower and upper case, numbers and special characters}, the head of 
department says, \enquote{so we have it too}.

What would you like to say in this conversation?

\begin{solution}
The last decades' research in user authentication says that such a policy yields 
bad security.
It forces users to select easy to guess passwords and incentivizes password 
reuse while more secure passwords are disqualified according to the policy.

A better policy is to have at least 12 characters as the only requirement.
Also, no requirement of updating the password at regular intervals --- only if a 
breach has occurred.
\end{solution}
\question[3]
  % tags: trustcomp:crypto:E:C:A
  Alice wants to provide confidentiality to a file.
  \begin{parts}
    \part She can accomplish this through mechanisms provided in the operating 
    system.
    Explain how this works and what the limits are.

    \part She can also accomplish this through purely cryptographic mechanisms.
    Explain how this works and what the limits are.
  \end{parts}

  \begin{solution}
    The first way she's securing her file is by using access control mechanisms 
    in the operating system (OS).

    Assuming we have physical access to the computer, then we can just read the 
    raw data from the disk.
    This can be accomplished by either booting our own OS on her computer, or 
    by removing the disk.

    If we don't have physical access we can always try to bypass the access 
    control mechanisms in other ways, e.g.\ by gaining privileges in the system 
    or seeing if the OS has failed to protect reading from the raw disk (i.e.\ 
    not using the file system).

    The main point here is that the operating system must be working correctly 
    for its mechanisms to be effective.
    The \emph{running} operating system will provide confidentiality by not 
    allowing other users' requests to open the file.

    The most obvious way to have system independent security for this file is 
    to encrypt it, i.e.~using cryptographic mechanisms.
    This way no one can read it unless they have access to the key, and this is 
    true no matter if you change the environment.
    (Of course, if the system is untrusted someone can get to the key that way, 
    but that's outside the scope of this question.)
  \end{solution}


  
\question[3]\label{q:software}
% tags: software:E:C:A
We have talked about how the users' mental models of how a program (and 
computer) works can endanger the users' security when the mental model and 
reality are not aligned.
This is true also for developers (we mentioned this when we talked about 
software security), give an example of how the developers' mental models are 
relevant for software security.

\begin{solution}
  Gollmann talked about broken abstractions.
  One example is characters: usually we abstract away the encoding and decoding 
  parts, we see them as characters and not bytes.
  So encodings like UTF-8 can cause problems since the same character can be 
  represented by several byte sequences.

  Another is the finite precision arithmetic that we work with in computers, 
  usually congruences modulo \(2^{32}\) or \(2^{64}\).
\end{solution}


\question[3]
% tags: sidechannels:E:C:A
Give an example of a side-channel attack and motivate why it is a side channel.
Explain whether your side-channel requires an active or passive adversary and 
discuss how difficult this side-channel will be to exploit.

\begin{solution}
  A side channel is an unintended channel emitting information which is due 
  to physical implementation flaws and not theoretical weaknesses or forcing 
  attempts.

  Extracting the secret key from a device by measuring energy consumption or 
  electromagnetic emissions while the device performs computations using the 
  secret key.

  This is a side channel since it relies on a weakness in the hardware 
  implementation of the crypto system.
  It is further an active attack since we might need the device to perform 
  operations on certain ciphertexts (or plaintexts).

  The example from the lecture of reading the electromagnetic emissions from a 
  screen display is a passive attack.
\end{solution}


