\documentclass[a4paper,11pt,addpoints]{miunexam}
\usepackage[T1]{fontenc}
\usepackage[utf8]{inputenc}
\usepackage[swedish]{babel}
\usepackage{url}
\usepackage{hyperref}
\usepackage{color}
\usepackage{varioref,prettyref}
\usepackage{SIunits}
\usepackage{pdfpages}
\usepackage{natbib}
\usepackage[today,nofancy]{svninfo}
\usepackage[varioref,prettyref,natbib]{miunmisc}

\svnInfo $Id$

\author{%
	Daniel Bosk\footnote{%
		E-post: \protect\url{daniel.bosk@miun.se},
		telefon: 060\,-14\,87\,09.
	} och
	Lennart Franked\footnote{%
		E-post: \protect\url{lennart.franked@miun.se},
		telefon: 060\,-\,14\,83\,86.
	}
}
\courseid{DT012G}
\course{Informationssäkerhet och riskanalys}
\date{2012-08-28}
\examtype{Tentamen}

%\printanswers

\begin{document}

\maketitle
\thispagestyle{foot}

\section*{Instruktioner}
\noindent
Läs uppgifterna noggrant innan du börjar att lösa dem.
Läs igenom samtliga uppgifter innan du börjar att lösa den första.
Planera din skrivning utifrån den tidsbegränsning som givits.
Besvara endast frågan, skriv inte svar som ej är relaterad till frågan.

Tänk på att skriva med ett korrekt språk, grammatik och meningsbyggnad är 
viktigt.
Svaret ska tydligt framgå.
Dina svar ska visa att du förstått, tänk på att utforma dem för att visa just 
dettae

Frågorna är \emph{ej} sorterade efter svårighetsgrad.

Lärare finns tillgänglig via telefon under tentan.
Svar på frågor som läraren anser berör samtliga kommer att publiceras i kursens
diskussionsforum.
Notera att diskussionsforumet ej kommer att övervakas under denna tentamen, 
ställ därför inga frågor här.

\begin{description}
	\item[Skrivtid] 30 maj 2012 kl. 13:00 till kl. 17:00.
		(4 timmar).
	\item[Hjälpmedel] Kurslitteratur, egna anteckningar och referensmaterial på
		webben.
	\item[Antal uppgifter] \numquestions
	\item[Antal poäng] \numpoints
\end{description}


\subsection*{Betygssättning}
\noindent
Denna tentamen betygsätts med betygen A, B och F.
För betygen E, D och C krävs inte denna tentamen, då räcker att du är godkänd 
på samtliga obligatoriska inlämningsuppgifter samt projekt.

Slutbetyget baserar sig på medelbetyget från de obligatoriska momenten.
Om du får ett B på denna tentamen ökas ditt slutbetyg med ett betygsteg.
På samma sätt om du får ett A ökas ditt slutbetyg med två betygsteg.

Preliminära betygsgränser för B är minst \unit{50}{\%} och för A krävs minst 
\unit{90}{\%}.


\clearpage
\section*{Uppgifter}
\noindent
Tentamen är uppdelad i delarna \emph{Terminologi} och \emph{Teori}.

\begin{questions}
	\uplevel{
		\subsection*{Terminologi}
		\noindent
	}
	\question\label{xrc:Terminologi}
		Förklara följande termer:
		\begin{parts}
			\part[1] NTLM
			\part[1] råstyrkemetoden (brute force)
			\part[1] ordlistemetoden (dictionary attack)
			\part[1] regnbågstabell (rainbow table)
			\part[1] spionprogram (spyware)
			\part[1] man-in-the-middle (MITM)
			\part[1] social engineering
			\part[1] Service Denial Attacks (SDA)
			\part[1] Distributed Denial of Service (DDoS)
			\part[1] hashvärde (hash value)
			\part[1] GAP-analys
			\part[1] LIS
			\part[1] säkerhetspolicy
			\part[1] nulägesanalys
			\part[1] FRAAP
			\part[1] ISO 27000
			\part[1] white hat
			\part[1] gray hat
			\part[1] black hat
			\part[1] sårbarhetsanalys
			\part[1] Advanced Persistent Threat (APT)
			\part[1] brandvägg
			\part[1] IDS
			\part[1] leetspeak
			\part[1] tvåfaktorsautentisiering (two-factor authentication)
			\part[1] phishing
		\end{parts}

	\uplevel{
		\subsection*{Teori}
		\noindent
	}

	\question[32]\label{xrc:Passwords}
		Beräkna och avgör genom olika avväganden vilken av följande 
		lösenordspolicyer du finner säkrast för hemmet,
		vilken du finner säkrast för en organisation med minst 500 medarbetare,
		samt vilken du finner säkrast för användning på webben.
		\begin{parts}
			\part Minst nio tecken beståendes av versaler, gemener och siffror.
            \part Minst åtta tecken beståendes av versaler, gemener, siffror och
            specialtecken (utgå från de tio som finns på siffrorna 0-9 på ett
            svenskt tangentbord)
			\part Tre slumpmässigt valda ord ur Svenska Akademins ordlista (SAOL) 
			\citep{SAOL}.
			SAOL innehåller cirka 125000 ord.
		\end{parts}
		Notera att avsaknad av motivering ger noll poäng, motivera väl!

	\question[16]\label{xrc:PDCA}
    Du inser när du börjar på företaget att det saknas ett ledningssystem för
    informationssäkerhet (LIS). Du går in till din chef, övertyga denna och
    ledningen att det behövs. Inled med att förklara vad det är, varför det
    behövs och hur det bör implementeras.

\end{questions}

\bibliography{literature}

\end{document}
