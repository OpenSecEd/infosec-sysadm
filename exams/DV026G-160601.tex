% $Id$
% Author:  Daniel Bosk <daniel.bosk@miun.se>
\documentclass[svv,addpoints]{miunexam}
\usepackage[utf8]{inputenc}
\usepackage[T1]{fontenc}
\usepackage[swedish,english]{babel}
\usepackage[hyphens]{url}
\usepackage{hyperref}
\usepackage{color}
\usepackage{prettyref,varioref}
\usepackage{subfigure}
\usepackage{amsmath,amssymb}
\usepackage{listings}
\usepackage{authblk}

\usepackage[autopunct]{csquotes}
\MakeBlockQuote{<}{|}{>}
\EnableQuotes{}

\usepackage[natbib,style=alphabetic,maxbibnames=99]{biblatex}

\printanswers{}

\examtype{Final exam}
\courseid{DV026G}
\course{Information Security}
\date{2016-06-01}
\author{%
  Daniel Bosk
}
\affil{%
  Department of Information and Communication Systems,\\
  Mid Sweden University, SE-851\,70 Sundsvall\\
  Email: \href{mailto:daniel.bosk@miun.se}{daniel.bosk@miun.se}\\
  Phone: 010-142\,8709
}

\DeclareMathOperator{\hmac}{HMAC}
\DeclareMathOperator{\xor}{\oplus}
\DeclareMathOperator{\concat}{||}

\begin{document}
\maketitle
\thispagestyle{foot}

\section*{Instructions}
\label{sec:Instructions}
Carefully read the questions before you start answering them.
Note the time limit of the exam and plan your answers accordingly.
Only answer the question, do not write about subjects remotely related to the
question.

Write your answers on separate sheets, not on the exam paper.
Only write on one side of the sheets.
Start each question on a new sheet.
Do not forget to \emph{motivate your answers.}

Make sure you write your answers clearly, if I cannot read an answer the answer
will be awarded no points---even if the answer is correct.
The questions are \emph{not} sorted by difficulty.

\begin{description}
  \item[Time] 5 hours.
  \item[Aids] Dictionary, course material and notes.
  \item[Maximum points] \numpoints{}
  \item[Questions] \numquestions{}
\end{description}

\subsection*{Preliminary grades}
The following grading criteria applies:
E \(\geq 50\%\),
D \(\geq 60\%\),
C \(\geq 70\%\), 
B \(\geq 80\%\),
A \(\geq 90\%\).
No question must be awarded zero points.


\clearpage
\section*{Questions}
The questions are given below.
They are not given in any particular order.

\begin{questions}
\question\label{q:crypto:foundations:E}
  Explain the following terms:
  \begin{parts}
    \part[1] Confidentiality
    \part[1] Integrity
    \part[1] Availability
    \part[1] Accountability
    \part[1] Non-Repudiation
  \end{parts}

  \begin{solution}
    See~\cite{Gollmann2011cs} and~\cite{Anderson2008sea} for definitions.
  \end{solution}


  
\question\label{q:usability:E:C:A}
  Human psychology is important in security.
  It is used in both security usability and social engineering.
  \begin{parts}
    \part[2] Give an overview of why psychology is important in 
    security.

    \begin{solution}
      Då systemen vi är beroende av och som ska upprätthålla vår säkerhet 
      handhas av människor blir psykologin genast viktig.
      Vi behöver psykologin inom säkerhetsområdet för att kunna ta hänsyn till 
      hur människor fungerar när vi konstruerar säkerhetssystem.
      Exempelvis, om vi gör ett system för komplext och användaren tycker att 
      komplexiteten är onödig, då kommer denne användare att aktivt försöka att 
      ta sig runt systemet --- kanske genom att skriva upp långa lösenord 
      istället för att lära sig dem utantill.

      Om vi däremot tar hänsyn till användarnas kognitiva begränsningar, då kan 
      vi konstruera system som både är säkra och enkla att använda.
    \end{solution}

    \part[4] Give an example of an attack which exploits weaknesses in human 
    psychology.
    Also explain why it works.

    \begin{solution}
      En psykologibaserad attack utnyttjar svagheter hos användarna för att ta 
      sig runt ett säkerhetssystem, det är alltså inte säkerhetssystemen som 
      angrips.

      Ett exempel på en sådan attack kan vara att en användare får ett e-brev 
      som till synes är från banken och som innehåller en länk till en 
      inloggningssida, kallat nätfiske.
      Brevet kan be användaren att uppdatera någonting hos banken via internet.
      Ett förfarande beskrivs och sedan läggs till \enquote{eller klicka på 
        länken}.
      Med en förfarande som låter som att det kan ta fem till tio klick kommer 
      användaren sannolikt att välja enklicksalternativet.
      Notera att förfarandet måste vara korrekt för banken medan länken är till 
      en phishingsida.
      Utformandet kan leda till vad litteraturen~\cite[s. 23]{Anderson2008sea} 
      kallar \emph{\foreignlanguage{english}{capture errors}}, att användaren 
      använder ett invant beteende: i detta fall att användaren klickar på 
      direktlänkar.

      Därutöver försöker nätfiskaren att få användaren att tillämpa fel regler 
      i situationen.
      Exempelvis, användaren kanske (omedvetet) lägger större vikt vid att ett 
      hänglås syns i webbläsaren för säker anslutning än att bankens namn är 
      rätt stavat i URL:en.
      Även att bankens namn finns med någonstans i URL:en kan vara en 
      tillräckligt stark regel för att användaren ska undvika att detektera den 
      felaktiga fiske-URL:en.
    \end{solution}
  \end{parts}


  
\question[2]\label{q:accountability}
What is the purpose of logging?

\begin{solution}
  The purpose of logging is to be able to follow how the system has 
  transitioned between states.
  We want to do this to be able to find vulnerabilities that might have been 
  exploited during a breach.
  Also to verify or reject possible breaches.
\end{solution}



\question[3]\label{q:software}
% examgen: software:A
Can a files such as images (e.g.\ JPEGs) and other data be dangerous?

\begin{solution}
  Yes, they can contain machine code which can be executed if there is e.g.\ 
  a buffer overrun vulnerability in the software that reads the data.
\end{solution}



\question\label{q:sidechannels}
% examgen: sidechannels:E:C
\begin{parts}
  \part[4] Give an example of a covert channel and
  \begin{solution}
    A server is anonymous (e.g.\ a Tor hidden service), i.e.\ you may access 
    the server but not know its location.
    Part of the server's service is giving the time.
    It has been shown that the variations in the system clock depend on the 
    ambient temperature.
    This means that by studying how the time on the server varies over day and 
    night and over the seasons, we can eventually figure out the ambient 
    temperature.
    From the ambient temperature we can later deduce the geographical location 
    of the server.
  \end{solution}

  \part[3] how we can prevent (or at least limit) it.
  \begin{solution}
    We can lower the resolution in the time-stamps the server gives, e.g.\ by 
    not giving seconds.
    This lowers the bandwidth of the covert channel, perhaps so that the attack 
    is infeasible.
    We could also sync the servers clock more often, e.g.\ by using the Network 
    Time Protocol.
    However, the only way to prevent it is by not revealing the time of the 
    server's system clock.
  \end{solution}
\end{parts}



\question[4]\label{q:trustcomp}
% examgen: trustcomp:E:C:A
Give an example of a DRM system, the idea behind it and why it works or not.

\begin{solution}
  Hardware dongles:
  You have a hardware dongle attached to the computer, the software can then 
  communicate with the dongle.
  The idea is that the software can be copied easily, but the dongle cannot.
  Thus the software can only run in as many instances as there are hardware 
  dongles.

  The hardware dongle can be simulated by other software in many cases.
  For the software to be able to tell the dongle and the simulated dongle 
  apart, it must be able to trust the operating system --- thus it needs to 
  verify the integrity of the operating system, which in turn requires special 
  hardware.
  The alternative approach is that the dongle is more sophisticated, e.g.\ that 
  it uses unforgeable digital signatures as output.
  In this case we instead modify the software itself, so that it simply skips 
  the checks with the dongle (e.g.\ the signature verification always returns 
  true).
\end{solution}



\question\label{q:passwd:infotheory:E:C:A}
  You are asked to estimate some password policies.
  The policies are the following:
  \begin{description}
    \item[basic12]
      At least 12 characters consisting of upper and lower case, and numbers.
    \item[randswedict4]
      Randomly choose four words from the Dictionary of the Swedish Language 
      (SAOL).
      This dictionary contains approximately 125\,000 words.
  \end{description}
  You should answer the following:
  \begin{parts}
    \part[4] Estimate the entropy for the passsword policies.
    (You may rely on the results in certain published research papers discussed 
    in the course for certain estimates.)

    \begin{solution}
      Basic12 requires at least 12 characters.
      We can use upper and lower case, as well as numbers.
      This yields an \emph{upper bound} of \(\log( 26+26+10 )\cdot 12\approx 
        71\) bits of entropy.

      Randswedict4 requires at least four words from the Swedish dictionary.
      This yield an \emph{upper bound} of \(\log( 125000 )\cdot 4\approx 68\) 
      bits of entropy.

      To achieve these upper bounds we must choose uniformly randomly.
      Most likely passwords under basic12 will yield an entropy somewhere 
      between that of basic8 and basic16 in~\cite{Komanduri2011opa}.
    \end{solution}

    \part[2] Decide how suitable they are for use in the home environment.
    \part[2] Decide how suitable they are for use in a web application.
  \end{parts}
  Note that you will not get any points without a motivation.


  
\question\label{q:passwd:auth:E}
  Explain the following terms:
  \begin{parts}
    \part[1] Brute force
    \part[1] Dictionary attack
    \part[1] Hash table
    \part[1] Social engineering
    \part[1] Two-factor authentication
    \part[1] Phishing
  \end{parts}


%  \%question\label{q:msb}
%  Explain the following terms:
%  \begin{parts}
%    \part[1] GAP-analys
%    \part[1] LIS
%    \part[1] säkerhetspolicy
%    \part[1] nulägesanalys
%    \part[1] ISO 27000
%  \end{parts}


  
\end{questions}


\printbibliography{}
\end{document}
