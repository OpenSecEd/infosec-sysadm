% $Id$
% Author:  Daniel Bosk <daniel.bosk@miun.se>
\documentclass[svv,addpoints]{miunexam}
\usepackage[utf8]{inputenc}
\usepackage[T1]{fontenc}
\usepackage[swedish,english]{babel}
\usepackage[hyphens]{url}
\usepackage{hyperref}
\usepackage{color}
\usepackage{prettyref,varioref}
\usepackage{subfigure}
\usepackage{amsmath,amssymb}
\usepackage{listings}
\usepackage{authblk}

\usepackage[autopunct]{csquotes}
\MakeBlockQuote{<}{|}{>}
\EnableQuotes{}

\usepackage[natbib,style=alphabetic,maxbibnames=99]{biblatex}

\printanswers{}

\examtype{Final exam}
\courseid{DV026G}
\course{Information Security}
\date{2016-08-16}
\author{%
  Daniel Bosk
}
\affil{%
  Department of Information and Communication Systems,\\
  Mid Sweden University, SE-851\,70 Sundsvall\\
  Email: \href{mailto:daniel.bosk@miun.se}{daniel.bosk@miun.se}\\
  Phone: 010-142\,8709
}

\DeclareMathOperator{\hmac}{HMAC}
\DeclareMathOperator{\xor}{\oplus}
\DeclareMathOperator{\concat}{||}

\begin{document}
\maketitle
\thispagestyle{foot}

\section*{Instructions}
\label{sec:Instructions}
Carefully read the questions before you start answering them.
Note the time limit of the exam and plan your answers accordingly.
Only answer the question, do not write about subjects remotely related to the
question.

Write your answers on separate sheets, not on the exam paper.
Only write on one side of the sheets.
Start each question on a new sheet.
Do not forget to \emph{motivate your answers.}

Make sure you write your answers clearly, if I cannot read an answer the answer
will be awarded no points---even if the answer is correct.
The questions are \emph{not} sorted by difficulty.

\begin{description}
  \item[Time] 5 hours.
  \item[Aids] Dictionary, course material and notes.
  \item[Maximum points] \numpoints{}
  \item[Questions] \numquestions{}
\end{description}

\subsection*{Preliminary grades}
The following grading criteria applies:
E \(\geq 50\%\),
D \(\geq 60\%\),
C \(\geq 70\%\), 
B \(\geq 80\%\),
A \(\geq 90\%\).
No question must be awarded zero points.


\clearpage
\section*{Questions}
The questions are given below.
They are not given in any particular order.

\begin{questions}
\question[3]\label{q:trustcomp}
% examgen: trustcomp:E:C:A
Describe the requirements for a process to be able to assess the integrity of 
itself and its execution environment.

\begin{solution}
  If the process can trust its environment (i.e.\ the operating system), then 
  it can rely on the environment to assess its own integrity.
  Thus the process relies on the integrity of the operating system.
  The oerating system in turn relies on the integrity of the hardware and must 
  rely on the hardware to assess its own integrity.
  Hence the process needs hardware that will not allow a modified version of 
  the operating system to run.
\end{solution}



\question\label{q:crypto:foundations:E}
  Explain the following terms:
  \begin{parts}
    \part[1] Confidentiality
    \part[1] Integrity
    \part[1] Availability
    \part[1] Accountability
    \part[1] Non-Repudiation
  \end{parts}

  \begin{solution}
    See~\cite{Gollmann2011cs} and~\cite{Anderson2008sea} for definitions.
  \end{solution}


  
\question\label{q:software}
% examgen: software:E:C
Describe the three malware reproduction techniques
\begin{parts}
  \part[1] virus,
  \begin{solution}
    The virus inserts its own code into other programs code.
    When the other programs are run the virus' payload is run too and the 
    infection can spread further.
  \end{solution}

  \part[1] worm,
  \begin{solution}
    The worm spreads by its own means, e.g.\ by utilizing networks (shared file 
    systems, remote executions bugs in network services) or emailing itself 
    using available programs on the computer.
  \end{solution}

  \part[1] trojan horse.
  \begin{solution}
    The trojan horse is a legitimate-looking program which contains unwanted 
    functionality.
    E.g.\ it is a flash-light app, but in the background it uploads the contact 
    list to the app's developer.
  \end{solution}
\end{parts}



\question\label{q:infotheory:passwd:E:C}
  Explain how information theory can be used to estimate the strength of 
  passwords chosen under a given password composition policy:
  \begin{parts}
    \part[2] How can you estimate the upper bound, i.e.~the maximum possible 
    entropy?

    \part[2] Why can't you estimate any (useful) lower bound, i.e.~the minimum 
    possible entropy?

    \part[2] How can you estimate the average case, i.e.~what is usually the 
    case when users choose the passwords?
  \end{parts}

  \begin{solution}
    You assume that every part of the password is chosen uniformly randomly.
    This gives the maximum entropy, i.e.~it is an upper bound.
    You have to account for all choices the password composition policy allows.
    Or rather, all choices the policy removes.

    This is hard because a user can choose a very easy to guess password, which 
    has almost no entropy.
    Similarly, if all users choose the same password, then the entropy would be 
    zero.

    The average case can be estimated as in~\cite{Komanduri2011opa}.
    You have to \emph{sample a lot of user-generated passwords}, then you can 
    perform a frequency analysis to find the probabilities and compute the 
    entropy.
    The users are the stochastic variable (random output) and you must get 
    a large enough sample to estimate the probability distribution.
  \end{solution}


  
\question[4]\label{q:sidechannels}
% examgen: sidechannels:E
Describe a scenario where a covert channel is used.

\begin{solution}
  A server is anonymous (e.g.\ a Tor hidden service), i.e.\ you may access the 
  server but not know its location.
  Part of the server's service is giving the time.
  It has been shown that the variations in the system clock depend on the 
  ambient temperature.
  This means that by studying how the time on the server varies over day and 
  night and over the seasons, we can eventually figure out the ambient 
  temperature.
  From the ambient temperature we can later deduce the geographical location of 
  the server.
\end{solution}



\question\label{q:passwd:auth:E}
  Explain the following terms:
  \begin{parts}
    \part[1] Brute force
    \part[1] Dictionary attack
    \part[1] Social engineering
    \part[1] Two-factor authentication
    \part[1] Phishing
  \end{parts}


 
\end{questions}


\printbibliography{}
\end{document}
