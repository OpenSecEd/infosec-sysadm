\question[3]
% tags: ac:auth:accountability:crypto:usability
% tags: E:C:A
You're having dinner with a few enthusiastic entrepreneurs, start-up starters 
and tech trendsters.
They want to create a revolutionizing food ordering app:
a user should be able to keep favourite orders (\eg custom pizzas) for easy 
reordering, easy but secure payments and delivery.

If the app doesn't have any security, it won't survive at all.
(If someone can order pizza at the expense of someone else, this is not going 
to work.)
You need great usability (to compete with the gazillion other food-ordering 
apps) and great privacy (also to compete with the gazillion other food-ordering 
apps).
(Well, nowadays, maybe the app won't survice the GDPR fines if it doesn't have 
privacy.)

What properties do you need for the different functions, what user data do they 
require?
(Remember: the principle of data minimization!)

Note: You don't have to rely on existing technologies, such as credit cards or 
Swish, you must specify the security and privacy properties you need for any 
function, \eg the payment system (that includes if you want to use Swish too).
There can be separate food and delivery services, they don't have to be the 
same, whatever you see fit to maximize security, privacy and usability.
Remember, you're working with some enthusiastic entrepreneurs who will not 
hesitate to create another trendy-tech start-up.

You're expected to use (\ie show) your knowledge and skills from the topics 
access control, authentication, accountability and usability.
You will also use (\ie show) your knowledge of some high-level properties from 
cryptography.
(Don't leave the exam room early, spend more time on this question instead, 
remember it's the size of 4--5 questions.)

\begin{solution}
  Keeping favourite orders \etc can be done by storing the data in the 
  smartphone app.
  That way it never leaves the user, hence it will not cause any problems.
  Then the user can back it up using the same means as for his or her other 
  data.

  For delivery, the address must be given to the delivery service, otherwise 
  they don't know where to deliver.

  The actual order, \eg pizzas and custom pizzas, must be given to the food 
  service, otherwise they don't know what to prepare.
  The food service can be different from the delivery service: the delivery 
  service only needs to know they deliver food (so they can keep it warm or 
  cold), the food service doesn't need to know where their products are 
  delivered.

  Both services must be compensated, the food service wants money for their 
  products, the delivery service wants money for the delivery.
  The food service shouldn't learn the cost of the delivery, that leaks some 
  information about the destination (\eg longer distance costs more).
  The delivery service shouldn't learn the price of the food, that leaks 
  information about the client (pricy food, wealthy client, good for robbery?).

  The payment system must support anonymous payments, \ie if Alice pays Bob 
  twice, Bob cannot link those payments together (and thus not link them to 
  Alice).
  It must also be secure, so that Alice cannot pay from Bob's account.

  The delivery service must deliver the right food to the right address.
  The food service must give the right food to the right delivery person.
  There must be an identifier, such as an order number, that is the same for 
  both.
  The delivery person must prove that he or she is authorized to pick up the 
  food (otherwise someone can say they're the delivery person although they're 
  not and simply steal the food).
  We can solve this as follows:
  When a user orders food, he or she will first book and pay the delivery 
  service.
  While doing this, he or she can give the delivery service a token.
  Then the user proceeds to pay the food at the food service, saying that the 
  person picking the food will present that token.
  This token will be hard to guess and will only be used once.
\end{solution}

\question[3]
  % tags: trustcomp:crypto:E:C:A
  Alice wants to provide confidentiality to a file.
  \begin{parts}
    \part She can accomplish this through mechanisms provided in the operating 
    system.
    Explain how this works and what the limits are.

    \part She can also accomplish this through purely cryptographic mechanisms.
    Explain how this works and what the limits are.
  \end{parts}

  \begin{solution}
    The first way she's securing her file is by using access control mechanisms 
    in the operating system (OS).

    Assuming we have physical access to the computer, then we can just read the 
    raw data from the disk.
    This can be accomplished by either booting our own OS on her computer, or 
    by removing the disk.

    If we don't have physical access we can always try to bypass the access 
    control mechanisms in other ways, e.g.\ by gaining privileges in the system 
    or seeing if the OS has failed to protect reading from the raw disk (i.e.\ 
    not using the file system).

    The main point here is that the operating system must be working correctly 
    for its mechanisms to be effective.
    The \emph{running} operating system will provide confidentiality by not 
    allowing other users' requests to open the file.

    The most obvious way to have system independent security for this file is 
    to encrypt it, i.e.~using cryptographic mechanisms.
    This way no one can read it unless they have access to the key, and this is 
    true no matter if you change the environment.
    (Of course, if the system is untrusted someone can get to the key that way, 
    but that's outside the scope of this question.)
  \end{solution}


  
\question[3]
% tags: infotheory:E:C:A
Browser fingerprinting attemtps to build a unique fingerprint for a web browser 
to track it across the web without making it store cookies.
Whenever a browser connects to a web server it gives some information to it, 
\eg browser make and model (\eg Firefox v59.0) which fonts are available \etc.
Explain how this can be used to create a unique fingerprint to track browsers 
(\ie users) across the web.
How much data is required?
(You don't have to give any numbers, just how to calculate them.)

\begin{solution}
  Browser versions vary, how often and when people update varies too.
  So there will be several browser versions used at the same time.
  On top of that, several browsers as well.
  Add a variety of fonts, plugins \etc.
  These can be combined in many ways.

  If there are many possible combinations, then the probability of two users 
  sharing the same setup is small.
  Thus these data yield a unique fingerprint to identity a user (\ie user's web 
  browser).

  We can see each property as a random variable and use entropy as a measure of 
  how much information they provide.
  We would then need the (base 2) logarithm of the number of people in the world 
  to uniquely track everyone.
\end{solution}
\question[3]
  % tags: foundations
  The terms \enquote{data} and \enquote{information} are related but not the 
  same.
  Discuss how they are related and how this affects the term 
  \enquote{security}, as in \enquote{data security} and \enquote{information 
  security}.

  \begin{solution}
  Data is an encoded representation of information.
  Data can be manipulated by formal rules to infer new information.
  E.g.\ the data \enquote{\(x+1 = 0\)} does not directly reveal the value of 
  \(x\), but we can manipulate the data with formal rules (equation solving) 
  and infer the value of \(x\) (\(x = -1\)).

  The same applies for other situations:
  When sending data over an anonymizing network (e.g.\ Tor), we only reveal the 
  size and timing of the packets sent.
  However, these can be statistically correlated to the packets exiting the 
  network, thus the sender and recipient can be inferred.
  \end{solution}
\question[3]
  % tags: passwd:infotheory:usability:E:C:A
  You are asked to analyse two password policies.
  The policies are the following:
  \begin{description}
    \item[basic12]
      Let \emph{the user choose} at least 12 characters consisting of: upper 
      and lower case, numbers.
    \item[randswedict4]
      \emph{Generate a password for the user} by randomly choosing four words 
      from the Dictionary of the Swedish Language (SAOL).
      This dictionary contains approximately 125\,000 words.
  \end{description}
  Analyse the two policies: what are the advantages and disadvantages of each, 
  how do they compare to each other.
\question[3]
% tags: software:passwd:E:C:A
Review the following piece of code.
Identify potential problems and explain why they are problems.

This is a shell script by the name of pwdauth, it runs with root priviledges.
\begin{lstlisting}
#!/bin/bash
# pwdauth <user>

# get the username from command line
username = $1

echo "Please enter the password for $username:"
read passwd

real_passwd = $(cat /secure/passwds | grep $username)

if [ $passwd = $real_passwd ]; then
  sudo -u $username /bin/bash
  exit 0
else
  exit 1
fi
\end{lstlisting}

\begin{solution}
  There are several problems.
  For example the username variable is not protected with quotation marks, 
  meaning it will be parsed.
  This means that commends and arguments can be injected as part of the 
  username.

  When getting the real password, one can pass a new file to search instead of 
  /secure/passwds.
  Also, one can set passwd to \verb'1 = 1 -o 1' to make the password check 
  always succeed.

  Since the program is running as root, one can get root priviledges by 
  exploiting this program.
\end{solution}

\question[2]\label{q:sidechannels}
% tags: sidechannels:E:C:A
Describe an attack scenario where a side-channel is of central interest.

\begin{solution}
  The adversary is interested in learning classified information.
  They set up a device which records electromagnetic emissions to reconstruct 
  the image on a screen, thus when a target works with the classified data on 
  the computer the adversary sees the same image.
  This is a passive attack.
\end{solution}

